\usepackage[utf8]{inputenc}
\usepackage[russian]{babel}
\usepackage{amsmath}
\usepackage{setspace}
\usepackage[a4paper, left=2.5cm, right=1.5cm, top=2cm, bottom=2cm]{geometry}
\usepackage{array}
\usepackage{ragged2e}
\usepackage{longtable}
\usepackage{blindtext}
\usepackage{titlesec}
\usepackage{tocloft} % Для изменения стилей оглавления
\usepackage{enumitem} % Для настройки списков
\usepackage{indentfirst}
\usepackage{graphicx}
\usepackage{caption}
%Отступ первой строки
\setlength{\parindent}{1.25cm}
%Расстояние между абзацами
\setlength{\parskip}{0pt}
%Междустрочный интервал
\linespread{1.5}

% Настройка формата для разделов
\titleformat{\section}[block]
{\normalfont\bfseries}
{\thesection} % Нумерация
{0.5em} % Отступ после номера 
{} % Допоолнительное оформления заголовкао
[] % Десйствие после заголовка


% Настройка формата для подразделов (subsection)
\titleformat{\subsection}[block]
{\normalfont\bfseries} % Жирный
{\thesubsection} % Нумерация подраздела
{0.5em} % Отступ между номером и заголовком
{}
[]

% Настройка формата для подподразделов (subsubsection)
\titleformat{\subsubsection}[block]
{\normalfont\bfseries} % Полужирный шрифт нормального размера
{\thesubsubsection} % Нумерация подподраздела
{0.5em} % Отступ
{}
[]

% Настройка стиля оглавления
\renewcommand{\cftsecfont}{\normalfont}
\renewcommand{\cftsecpagefont}{\normalfont}
\renewcommand{\cftsecleader}{\cftdotfill{\cftdotsep}}
\renewcommand{\cfttoctitlefont}{\normalsize }
\addto\captionsrussian{\renewcommand{\contentsname}{СОДЕРЖАНИЕ}}
\setlength{\cftbeforesecskip}{0cm}

\captionsetup[figure]{name=Рисунок, labelsep=space}