\newpage
\section{СОЗДАНИЕ ПРИМИТИВНОГО ПРОЕКТА С НУЛЯ}

\subsection{Шаг 1 Project Browser}
Первым делом для создания проекта нам потребуется открыть unreal project browser.
Unreal Project Browser представляет собой конфигуратор для создания проектов.
В нем вы можете увидеть разные шаблоны(Шутер от первого лица, Игра с видом сверху и тд.)

Но мы выберем пустой(Blank) шаблон с такими конфигурациями:
\begin{figure}[h]
    \centering
    \includegraphics*[width = \textwidth]{Lections/ProjectBrowser.png}
\end{figure}

\subsection{Шаг 2 Настройка интерфейса и cоздание yровня}
Для начала установим классический интерфейс редактора через верхнее меню \textbf{Window->Layout->UE4ClassicLayout}

\begin{figure}[h]
    \centering
    \includegraphics*[width = \textwidth]{Lections/Layout.png}
\end{figure}


На старте нам дается уровень с большой картой. Давайте создадим новый уровень с упрощенной картой.

В верхнем меню переходим \textbf{File->New Level->Basic} и нажимаем create. Теперь нужно сохранить эту карту в папку проекта с помощью сочетания клавиш \textbf{Ctrl-S}. Для удобства лучше выделить отдельную папку \textbf{Levels} и сохранить новый уровень туда с припиской \textbf{L\_}.


\begin{figure}[h]
    \centering
    \includegraphics*[width = \textwidth]{Lections/LevelChange.jpg}
\end{figure}

\newpage

\subsection{Шаг 3 Работа с базовыми объектами}

В классическом отображении интерфейса слева вы можете видеть панель \textbf{Place Actors}

\textbf{Actor} -- это любой объект который может быть расположен на уровне(\textbf{Level}).

\textbf{Actor} состоит из компонентов и может влиять на их работу.

\textbf{Components}(компоненты) -- это определенные объекты с определенной функциональностью которые мы можем добавить к \textbf{Actor} для расширения его возможностей. 

Тем самым Unreal Engine поддерживает вложенно-агрегированную\footnote{Агрегация это способ размещения объектов внутри друг друга при котором вложенные объекты и основной не зависимы друг от друга.} структуруру позволяя тасовать объекты внутри друг друга меняя на лету функционал и поведение Actor`ов.

Основным примером данной структуры является отношение уровня к actor`ам размещенных в нем.
То есть сам по себе уровень может существовать в не зависимости от расположенных в нем artor`ов.


\begin{figure}[h]
    \centering
    \includegraphics*[width = \textwidth]{Lections/LevelAgrigation.png}
\end{figure}

Для примера разместим на уровне cube из меню Place Actors. Теперь с помощью горячих клавиш(W,E,R) или кнопок viewport можно переключатся между видами изменения трансформации(Transform) объекта.

\textbf{Transform} - это структура содержащая 3 вектора(Location - расположение, Rotation - поворот, Scale - размерность) определяющая трансформацию объекта в пространстве.

Для изменения других свойств данного actor`а можно обратиться к меню \textbf{Details}.

Меню \textbf{Details} -- Здесь собраны параметры actor`a и его компонентов расположенного на сцене. 

Поскольку у компонента StaticMeshComponent есть возможность включения физики. Включим эту возможность и поднимем куб над полом. 
Теперь запустим проект в режиме simulate.

\begin{figure}[h]
    \centering
    \includegraphics*[width = \textwidth]{Lections/image.png}
\end{figure}

\subsection{Шаг 4 Режимы запуска проекта}



Для тестирования и отладки проекта существует несколько вариантов запуска игры.

1.\hspace{1em}Selected Viewport -- классический запуск игры в окне viewport.

2.\hspace{1em}New Editor Window -- запуск игры в отдельном окне редактора. 

3.\hspace{1em}VR Preview -- запуск в очках виртуальной реальности.

4.\hspace{1em}Standalone Game или Mobile -- Запуск в отдельном процессе.

5.\hspace{1em}Simulate -- Запуск в режиме наблюдателя.

\subsection{Шаг 5 Базовая логика и level blueprint}

